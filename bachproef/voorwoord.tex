%%=============================================================================
%% Voorwoord
%%=============================================================================

\chapter*{\IfLanguageName{dutch}{Woord vooraf}{Preface}}
\label{ch:voorwoord}

%% TODO:
%% Het voorwoord is het enige deel van de bachelorproef waar je vanuit je
%% eigen standpunt (``ik-vorm'') mag schrijven. Je kan hier bv. motiveren
%% waarom jij het onderwerp wil bespreken.
%% Vergeet ook niet te bedanken wie je geholpen/gesteund/... heeft

Als ontwikkelaar vind ik het uitermate belangrijk om een goede kennis te hebben van waar je mee aan het werken bent, van het project zelf tot de individuele packages en libraries die je in gebruik neemt. React interesseerde me van de dag dat ik ervan hoorde, en na er een maand in een professionele omgeving mee gewerkt te hebben heb ik een hoog respect ontwikkeld voor hun ontwikkelaarsteam.

Ik heb al meerdere keren met Webpack te maken gehad, en de configuratie is altijd al verwarrend en vervelend geweest in mijn ervaring. Toen ik hoorde over Vite zijn performantie en configuratie had het direct mijn volle aandacht. Ik ben heel benieuwd om te zien welke alternatieven ook nog bestaan, en wat voor extra's ze klaar hebben staan.

\vspace{2cm}

Mijn broer, Pieter-Jan, die vóór mij Toegepaste Informatica gevolgd heeft aan HoGent. Ik herinner me nog toen ik 16 was en wou leren programmeren dat hij me met plezier de basis had geleerd van Java. Zonder de steun die ik toen kreeg, en tot op heden toe nog steeds krijg zou ik waarschijnlijk niet zijn waar ik nu ben in mijn leven.
